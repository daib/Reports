%%This is a very basic article template.
%%There is just one section and two subsections.
\documentclass[12pt]{article}
\usepackage{amsmath}
\usepackage{amssymb}
\usepackage{graphicx}
\usepackage[colorlinks]{hyperref}
\usepackage{cite}
\usepackage{setspace}
\doublespacing

\addtolength{\oddsidemargin}{-.7in}
% \addtolength{\evensidemargin}{-2in}
\addtolength{\textwidth}{1.5in}
% 
\addtolength{\topmargin}{-1.2in}
\addtolength{\textheight}{2.3in}

\begin{document}


\title{Analyse and Prediction of Package Timing Patterns of Streaming
Applications on Multicore Machines}
\author{Dai Bui}
\maketitle
\section{Introduction}
Multicore machines have become more and more popular recently as a mechanism to
speed up applications while increasing clock frequency has become more and more
challenging. Now, applications are parallelized, partitioned and mapped into
several difference processors instead of running on a single processor. However, energy
saving is also one important aspect of the future computing research as there
are more and more computing devices produced each year that would consume a great
amount of energy. Streaming applications, i.e. MP3, MPEG4, e.t.c, is an
important class of applications, however, are not very difficult to parallelize.

In this project, we aim at using time series analsysis
techniques~\cite{BrillingerTimeSeries, ShumwayTimeSeries} to analyse and predict
timing interval patterns of packages sent between partitions of a streaming
applications on a multicore machine.

\section{System Model}
\subsection{System Architecture}
We assume a message passing multicore system composed of several processing
cores connected using a network on chip. Processing core (CPU) has its own
memory and is connected a router using a network interface (NI) as in
Figure~\ref{fig:MulticoreNoC}.

\begin{figure}[ht!]
\centering
\includegraphics[width=0.7\columnwidth]{img/MulticoreNoC}
\caption{System architecture}\label{fig:MulticoreNoC}
\end{figure}

\subsection{Synchronous Dataflow Model of Computation}
One of the important feature of streaming applications that makes it possible to
predict the timing interval patterns of packages between partition is the
\textit{periodicity} semantic based on synchronous dataflow (SDF)~\cite{LeeSDF}
model of computation. Figure~\ref{fig:SDF} shows a graph of a simple SDF
application.

\begin{figure}[ht!]
\centering
\includegraphics[width=0.5\columnwidth]{img/SDF}
\caption{An graph of an SDF application}\label{fig:SDF}
\end{figure}

In Figure~\ref{fig:SDF}, the SDF application is composed of three processes
called \textit{actors}. Each actor has multiple input and output ports. Each
input/output port is annotated with a number which is the number of
\textit{tokens} the port will consume/produce each time the actor containing the
port executes, so-called ~\textit{fires}. A token is a unit of data, it could be
a number, a package, e.t.c.

\subsubsection{SDF Applications Partitioning}
An SDF application is often composed of several actors and is partitioned into
several parts and each part is then mapped to a core in a multi-core machine.
For example, in Figure~\ref{fig:SDF}, an SDF application can be partitioned into
two parts in which the first part is composed of actors $A$ and $B$, the second
part is composed of actor $C$.

\subsubsection{Data Accumulation}
Data sent by actors located inside a core to another actor in another core are
first put into a buffer. When the buffer is full, data is sent to the
destination core. This accumulation mechanism could help increase the
\textit{silent} intervals between packages and thus reducing the number of time
to turn-on/off links and saving more power as turning links on/off consumes some
amount of power.


\section{Experiment}
\subsection{System Setup}
We implemented a simulation framework of a distributed memory multicore using
the SuperH processors~\cite{Stanley-MarbellSunflower} connected using a
cycle accurate Wormsim~\cite{Wormsim} network on chip simulator with Orion power
model~\cite{WangOrion}. We use Streamit benchmarks~\cite{ThiesStreamIt} that
respect SDF semantics to run on our simulator to obtain message trace and power
saving data.

\subsection{Package Timing Trace}

To evaluate intervals between packages, we first run our application to obtain a
network trace of messages. Based on that we use time series analysis
techniques~\cite{BrillingerTimeSeries, ShumwayTimeSeries} to build a statistical
model of messages based on that trace. The statistical model is then used to
predict the future message intervals.
\section{Timing Analysis}
\subsection{ACF}
\subsection{ARMA}
\subsubsection{Problem with normal ARMA}
\subsubsection{Seasonal ARMA}

\subsection{Correlation of Package Send/Receive Time}

\bibliographystyle{plain}
\bibliography{PkgTiming}

\end{document}
