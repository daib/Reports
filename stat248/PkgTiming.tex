%%This is a very basic article template.
%%There is just one section and two subsections.
\documentclass[12pt]{article}
\usepackage{amsmath}
\usepackage{amssymb}
\usepackage{graphicx}
\usepackage[colorlinks]{hyperref}
\usepackage{cite}
\usepackage{setspace}
\doublespacing

\addtolength{\oddsidemargin}{-1in}
\addtolength{\evensidemargin}{-.875in}
\addtolength{\textwidth}{2in}
% 
\addtolength{\topmargin}{-1.2in}
\addtolength{\textheight}{2.3in}

\begin{document}


\title{Evaluation and Prediction of Package Timing of Streaming Applications on
Multicore Machines}
\author{Dai Bui}
\maketitle
\section{Introduction}
Streaming applications' performance has been recognized as being
predictable~\cite{AleenInputDriven, PoplavkoExecutionStreams,
Theelen06ScenarioAware}. This predictable feature of streaming applications
lies in the \textit{periodicity} of their underlying model of computation
semantics~\cite{LeeSDF, ThiesStreamIt, GlitiaArrayOLDelay}. The inherent
parallelism of streaming languages makes it suitable for multicore machines.

The emerging Network on-Chip (NoC) interconnection
paradigm~\cite{DallyPacketNotWire, demicheli_noc_book_2006} has been an
interesting replacement for the traditional bus system which is deemed to be
unscalable for large multicore machines with hundreds of cores.

However, the route diversity advantage of NoC that makes it scalable can also
significantly increase the power consumption of a multicore machine. One of the
potential power saving scheme is to turn off some links when they are not used.
However, in order to do that, we need to know the timing of traffic between
nodes, which is difficult for general applications. However, for streaming
applications with their periodic traffic nature allow us to estimate timing of
traffic between processing nodes using statistics methods such as time series
analysis.

 
\section{Synchronous Dataflow}
\section{Architectural and Communication Models}
\subsection{Multicore and Network Models}
We assume a message passing multicore system composed of several processing
cores connected using a network on chip. Processing core has its own memory and
is connected a router using a network interface.

\subsection{Communication Mechanism}
\subsection{SDF Applications Partitioning}
An SDF application is often composed of several actors and is partitioned into
several parts and each part is then mapped to a core in a multi-core machine.
For example, in Figure~\ref{}, an SDF application is partitioned into three
parts and then mapped to three different cores.

\subsubsection{Data Accumulation}
Data sent by actors located inside a core to another actor in another core are
first put into a buffer. When the buffer is full, data is sent to the
destination core. This accumulation mechanism could help increase the
\textit{silent} intervals between packages and thus reducing the number of time
to turn-on/off links and saving more power as turning links on/off consumes some
amount of power.

\subsubsection{Active Message Communication}

\subsubsection{Credi-based Synchronization}
To avoid the messaged dependent deadlock caused by unconsumed packages at
destination nodes, we employ a credit based mechanism in which each
communication flow between a source and a desination has a number of buffers of
a specified size at the destination. Whenever the sender fills up one buffer at
the receiver, it decreases its credit counter by one and will stop sending data
when the credit counter reachs zero. Whenever the receiver consumes (process) a
buffer, it sends back a credit the the sender so that the sender knows that the
is one more free buffer at the destination and it can resume sending data.

\section{Package Interval Prediction}
\subsection{Package Trace}
To evaluate intervals between packages, we first run our application to obtain a
network trace of messages. Based on that we use time series analysis
techniques~\cite{} to build a statistical model of messages based on that trace.
the statistical model is then used to predict the future message intervals.

\subsection{ARMA}

\section{Experiment}
We implemented a simulation framework of a distributed memory multicore using
the SuperH processors~\cite{Stanley-MarbellSunflower} connected using a
cycle accurate Wormsim~\cite{Wormsim} network on chip simulator with Orion power
model~\cite{WangOrion}. We use Streamit benchmarks~\cite{ThiesStreamIt} that
respect SDF semantics to run on our simulator to obtain message trace and power
saving data.

\section{Related Work}
There are several works on power optimizations of links on network on
chips~\cite{LiCompilerDirected, LiProfileDriven,MuralidharaCommBased} for
general applications. However, it is difficult to evaluate the interval of
packages sent in a general applications as the control flow of a general
applications can be complicated.

\bibliographystyle{plain}
\bibliography{PkgTiming}

\end{document}
