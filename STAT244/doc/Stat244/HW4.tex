%%This is a very basic article template.
%%There is just one section and two subsections.
\documentclass{article}

\usepackage{amsmath}
\usepackage{amssymb}
\usepackage{graphicx}
\usepackage[colorlinks]{hyperref}
\usepackage{listings}
\usepackage{fancyvrb }
\usepackage{subfigure}


\addtolength{\oddsidemargin}{-.875in}
\addtolength{\evensidemargin}{-.875in}
\addtolength{\textwidth}{1.75in}

\addtolength{\topmargin}{-.875in}
\addtolength{\textheight}{1.75in}

\newenvironment{mylisting}
 {\begin{list}{}{\setlength{\leftmargin}{1em}}\item\scriptsize\bfseries}
 {\end{list}}

 \newenvironment{mytinylisting}
 {\begin{list}{}{\setlength{\leftmargin}{1em}}\item\tiny\bfseries}
 {\end{list}}


\title{STAT244 - Statistical Computing\\Homework 4}
\author{Dai Bui}
\begin{document}

\date{}
\maketitle
\section{Binary Tree}
\subsection{Overview}
In this problem, we implement a simple (non self-balancing) binary tree and
compare the execution time with a self-balancing binary tree based on a random
1200 integer numbers input.

Binary tree data structure can help finding data faster than normal linked list
data structure. However, the speed of inserting and finding data in a binary
tree also depends on the input data that form the run-time structure of the
tree. To evaluate this effect, we will compare the execution time of two
implementations, a normal (not self-balancing tree) and a self-balancing tree.
We test those implementations with a random input of 1200 integer numbers in
both sorted and unsorted fashions.


\subsection{Implementation}
We implemented a program with source code listed in
Appendix~\ref{sec:bintreeCode}. The program uses Linux real-time clock library
to measure the execution time in nano-second. The output of the
program for a data sequence of 1200 random integer number listed in
Appendix~\ref{eqn:bintreeInputfile} is as follows:
\VerbatimInput[baselinestretch=1,fontsize=\footnotesize,numbers=left]{hw4files/bintreeoutput.txt}

\subsection{Conclusion}
As we can see that, a non-self-balancing tree performs better than
a self-balancing tree  on a unsorted sequence. This is because a self-balancing
tree has to spend time keeping its runtime structure balanced.

However, for a sorted sequence, a non-self-balancing tree performs rather bad.
It is 10 times slower than a balancing tree on the same sorted sequence. This
is because a sorted sequence will cause a non-self-balancing tree to become a
simple linked list, there for it takes more time to traverse the tree to search
for suitable node at the end of the tree (as this is a sorted sequence).
Meanwhile, the performance of a self-balancing tree is improved over its
performance on a unsorted sequence. This is perhaps because it takes a
self-balancing tree less time to maintain its balanced structure on a sorted
sequence than on an unsorted sequence.

As we can see that keeping a tree balanced is very important. Then since, a
non-self-balancing tree performs better than a self-balancing tree  on a
unsorted sequence. We can see that a random unsorted sequence can keep a
tree pretty balanced.

As there are many duplicated numbers in the input sequence, our program has
detected those duplications but we do not list that output here as it takes
lots of space.

\section{Analysis of Variance - ANOVA}
\subsection{Overview}
This problem exposes me to a new aspect that I was not used to called data
analysis~\cite{MasonDataAnalysis, KeppelDataAnalysis}. Basically, when there are
several factors with multiple levels contributing to one output result, we
would like to analyse and measure the main effect of each factor as well as the
joint effect between factors.

In this problem, we implement a \textit{balanced} multi-factor analysis program.
The details of the analysis technique is described in several sources such
as Chaper 6 in ~\cite{MasonDataAnalysis} and Chapter
13 in~\cite{KeppelDataAnalysis}. Page 179 in~\cite{MasonDataAnalysis} has a
clear summary of the computation methods for three factors, see
Figure~\ref{fig:threeFactors}. From that, we are able to generalize the
computation method for more factors.

 \begin{figure}[ht!]
\centering
\includegraphics[width=0.95\columnwidth]{hw4files/threeFactorAnalysis.png}
\caption{Computing method for three-factor analysis.}
\label{fig:threeFactors}
\end{figure}

\subsection{Implementation}
We implemented a program using a self-balancing binary tree to store data
structure. Our program can theoretically handle an infinite number of factors.
Users can also limit the number of factors used in the analysis. We test our
program on a data input file listed in Appendix~\ref{sec:dataANOVA}. We also
used \texttt{dcdflib} to find the F probability. When doing this problem, I
learned one interesting technique to use a powerset of those factors to generate
all possible joint effects of those factors.

\subsection{Results and Conclusion}
The output of our program for the input data in Appendix~\ref{sec:dataANOVA} is
the same as the output from SAS for analysis of both three and two factors.
\subsubsection{Three-factor Analysis}
Following is the output of the program for three factors.
\VerbatimInput[baselinestretch=1,fontsize=\footnotesize,numbers=left]{hw4files/threeFactors}

which is numerically similar to that of SAS.
\VerbatimInput[baselinestretch=1,fontsize=\footnotesize,numbers=left]{hw4files/a3.txt}
\subsubsection{Limiting to Two-factor Analysis}
For the same data input, we do ANOVA for only 2 first factors and get the
result:
\VerbatimInput[baselinestretch=1,fontsize=\footnotesize,numbers=left]{hw4files/twoFactors}

which is again numerically similar to that of SAS
\VerbatimInput[baselinestretch=1,fontsize=\footnotesize,numbers=left]{hw4files/a3_2.txt}

Doing this problem helps me learn much more about data analysis. However, my
program only works for balanced data. For non-balenced data, the result is
different from that of SAS as we have not done any fiting of data.
The source code of our ANOVA program is listed in Appendix~\ref{sec:ANOVAProg}.

\bibliographystyle{plain}
\bibliography{Stat}

\appendix
\section{Binary tree source code}\label{sec:bintreeCode}
\subsection{main.c}
\VerbatimInput[baselinestretch=1,fontsize=\footnotesize,numbers=left]{hw4files/main.c}
\subsection{btree1.c and tree1.h}
The same as files from the class website.
\subsection{timer.h}
\VerbatimInput[baselinestretch=1,fontsize=\footnotesize,numbers=left]{hw4files/timer.h}
\subsection{timing.h}
\VerbatimInput[baselinestretch=1,fontsize=\footnotesize,numbers=left]{hw4files/timing.h}
\subsection{timing.c}
\VerbatimInput[baselinestretch=1,fontsize=\footnotesize,numbers=left]{hw4files/timing.c}
\subsection{loadmat.c}\label{sec:appLoadmat}
\VerbatimInput[baselinestretch=1,fontsize=\footnotesize,numbers=left]{hw4files/loadmat.c_1}
\subsection{Makefile}
\VerbatimInput[baselinestretch=1,fontsize=\footnotesize,numbers=left]{hw4files/Makefile_1}
\subsection{Data input file}\label{eqn:bintreeInputfile}
\VerbatimInput[baselinestretch=1,fontsize=\footnotesize,numbers=left]{hw4files/data}


\section{Source code for ANOVA program}\label{sec:ANOVAProg}
\subsection{anova.c}
\VerbatimInput[baselinestretch=1,fontsize=\footnotesize,numbers=left]{hw4files/anova.c}
\subsection{tree1.h}
\VerbatimInput[baselinestretch=1,fontsize=\footnotesize,numbers=left]{hw4files/tree1.h}

\subsection{btree1.c}
\VerbatimInput[baselinestretch=1,fontsize=\footnotesize,numbers=left]{hw4files/btree1.c}
\subsection{loadmat.c}
The same as the file in Appendix~\ref{sec:appLoadmat}.
\subsection{Makefile}
\VerbatimInput[baselinestretch=1,fontsize=\footnotesize,numbers=left]{hw4files/Makefile}
\subsection{dcdflib.c, ipmpar.c and cdflib.h}
Those dcdflib files are available at the
\href{http://www.netlib.org/random/dcdflib.c.tar.gz}{link}.

\subsection{Data input file}\label{sec:dataANOVA}
\VerbatimInput[baselinestretch=1,fontsize=\footnotesize,numbers=left]{hw4files/d3}
\end{document}

