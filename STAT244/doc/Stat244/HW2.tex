%%This is a very basic article template.
%%There is just one section and two subsections.
\documentclass{article}

\usepackage{amsmath}
\usepackage{amssymb}
\usepackage{graphicx}
\usepackage[colorlinks]{hyperref}
\usepackage{listings}
\usepackage{fancyvrb }
\usepackage{subfigure}


\addtolength{\oddsidemargin}{-.875in}
\addtolength{\evensidemargin}{-.875in}
\addtolength{\textwidth}{1.75in}

\addtolength{\topmargin}{-.875in}
\addtolength{\textheight}{1.75in}

\newenvironment{mylisting}
 {\begin{list}{}{\setlength{\leftmargin}{1em}}\item\scriptsize\bfseries}
 {\end{list}}

 \newenvironment{mytinylisting}
 {\begin{list}{}{\setlength{\leftmargin}{1em}}\item\tiny\bfseries}
 {\end{list}}


\title{STAT244 - Statistical Computing\\ Homework 2}
\author{Dai Bui}
\begin{document}
\date{}
\maketitle
\section{Dynamic Graphics}
\subsection{Overview}
This problem explores the effects of window width on
smoothing. Smoothing techniques have several applications such as explorary data
analysis, model building, goodness-of-fit, parametric estimation,
etc.~\cite{SimonoffSmoothingTechs}. The smoothing method is a compromise
between the parametric modeling, which needs to assume the underlying model of
empirical data, and a purely non-parametric modeling, which does not make any
assumptions about the model. The parametric approach may lead to misleading
interpretations of observed data when the assumption about the data model is
wrong, while purely non-parametric approach does not provide any true summary
of data.

Let $\{x_1, x_2, \ldots, x_n\}$ be a random sample set of size $n$ from the
variable $X$ that has probability density function $f(x)$. Since $f$ is not
known, we want to find an estimate $\hat{f}$ for $f$ from the empirical
observations. 

\begin{equation}\label{eqn:kernalDensityEstimation}
\hat{f}=\frac{1}{nh}\sum_{i=1}^nK\bigl(\frac{x-x_i}{h}\bigr)
\end{equation}

 The equation~\ref{eqn:kernalDensityEstimation} is form of a \textit{kernel
 density estimator} with kernel function $K$. There are several kernel functions
 such as: gaussian, rectangular, triangular, epanechnikov, biweight, cosine,
 optcosine. In the equation~\ref{eqn:kernalDensityEstimation}, $h$ is bin width, so-called bandwidth. When we change the type of the kernel function $K$, and/or the bin
 width $h$, we will get different shapes of the estimate $\hat{f}$.
 
 \subsection{Implementation}
 We implemented a program in R to study the effects of bin width on the
 smoothness of the estimate density function.
 
 When an user first run the program, an open file dialog window appears for the
 user to browse and select a file to open as in Figure~\ref{fig:openFileDlg}.
 For example, we use the data file
 \href{http://pages.stern.nyu.edu/~jsimonof/SmoothMeth/Data/Tab/cdrate.tab}{cdrate.tab} 
 used in the book~\cite{SimonoffSmoothingTechs}. As the data file can has more
 than one column, an user needs to select which column they want to use for
 the smoothing program to draw. We do this by open a dialog with radio buttons
 for the names of data columns as in Figure~\ref{fig:dataColSelection}.
 
 \begin{figure}[ht!]
\centering
\includegraphics[width=0.5\columnwidth]{hw2files/openFileDlg.png}
\caption{Open file dialog window.}
\label{fig:openFileDlg}
\end{figure}
 
\begin{figure}[ht!]
\centering
\includegraphics[width=0.3\columnwidth]{hw2files/dataColSelection.png}
\caption{Data column selection.}
\label{fig:dataColSelection}
\end{figure}
 
 After that, the main window of the program appears as in
 Figure~\ref{fig:smoothingMainWindow}. The main window has a slider to vary the
 \textit{number of bins}, thereby varying the bin width (bandwidth) $h$ of the
 smoothing method. There are radio buttons on the main window for users to
 select which type of kernel function $K$ to use in the kernel density
 estimator. By default, the kernel function is Gaussian.
 
 \begin{figure}[ht!]
\centering
\includegraphics[width=0.7\columnwidth]{hw2files/smoothingProgram.png}
\caption{Main window of the smoothing program.}
\label{fig:smoothingMainWindow}
\end{figure}

Users can open a new data file by clicking on the \texttt{Open data file}
button on the main window.
 
 We choose to use the number of bins instead of the bin width in the slider,
 however, the bin width can be derived from the number of bins as follows:
 \begin{equation}
 h= \frac{dataRange}{\#\ of\ bins}
 \end{equation}
 This bin width is used to compute the density estimator by using the
 \texttt{density} function in R.
 
 One corner case when we open a data file, users might select a non-parametric
 data column. In that case, we will display a message box telling users that
 the data column cannot be used in smoothing methods.
 \subsection{Conclusion}
 For experiment the smoothness, we use the data set available online
 \href{http://pages.stern.nyu.edu/~jsimonof/SmoothMeth/Data/Tab/elusage.tab}{elusage.tab}
 from the same book~\cite{SimonoffSmoothingTechs}.
 
 Figures~\ref{fig:gaussianKernel} and ~\ref{fig:rectangularKernel} show that Gaussian kernel function is smoother than rectangular kernel. From
 Figures~\ref{fig:overSmoothing}, ~\ref{fig:underSmoothing} and
 ~\ref{fig:gaussianKernel}, we can see that when the bin width is too big, the
 density estimator is too smooth which might over assume about the data model.
 This bias toward the kernel function $K$ model may lead to misleading
 interpretation about the data. When the bin width is too small, the density
 estimator does not capture and predict the data tendency quite well as in
 Figure~\ref{fig:underSmoothing} as it is too rough. When the bin width is not
 too big or too small as in Figure~\ref{fig:gaussianKernel}, the
 density estimator can capture the data model quite well.
 
  \begin{figure*}[ht!]
\centering
\subfigure[Over-smoothing when the bin width is too large.]{
\includegraphics[width=0.47\textwidth]{hw2files/UsageBin2G.png}
\label{fig:overSmoothing}
}
\subfigure[Under-smoothing when the bin width is too small.]{
\includegraphics[width=0.47\textwidth]{hw2files/UsageBin100G.png}
\label{fig:underSmoothing}
}
\subfigure[Smoothing with Gaussian kernel function.]{
\includegraphics[width=0.47\textwidth]{hw2files/UsageBin32G.png}
\label{fig:gaussianKernel}
}
\subfigure[Smoothing with Rectangular kernel function.]{
\includegraphics[width=0.47\textwidth]{hw2files/UsageBin32Rec.png}
\label{fig:rectangularKernel}
}

\caption{Smoothness}
\end{figure*}

 
 I like this problem as it exposes me to a new aspect that I did not know
 before. Learning using R with Tcl/Tk is quite fun.
 
 Finally, the source code in R of the program is listed in
 Appendix~\ref{sec:appSmoothing}.
 
\bibliographystyle{plain}
\bibliography{Stat}
\appendix
\section{Source code for smoothing program in R}\label{sec:appSmoothing}
\VerbatimInput[baselinestretch=1,fontsize=\footnotesize,numbers=left]{hw2files/smoothing.R}


\end{document}
